\documentclass[a4paper,14pt]{extarticle}
\usepackage{rotating}
\usepackage{verbatim}
%%% Преамбула %%%

\usepackage{fontspec}
\usepackage{xunicode}
\usepackage{xltxtra}
\usepackage{pdfpages}

%%\usepackage[europeanresistors,americaninductors,
%%            americancurrents,americanvoltage]{circuitikz}



%
% Шрифты
%
\defaultfontfeatures{Ligatures=TeX}
\newfontfamily{\cyrillicfont}{Tinos}
\newfontfamily{\cyrillicfontrm}{Tinos}
\newfontfamily{\cyrillicfonttt}{Cousine}
\newfontfamily{\cyrillicfontsf}{Arimo}
\setmonofont{Cousine}[Scale=MatchLowercase]

% Красная строка после заголовка
%%\usepackage{indentfirst}

% Русский язык
\usepackage{polyglossia}
\setdefaultlanguage{russian}

\usepackage{amssymb,amsfonts,amsmath} % Математика
\numberwithin{equation}{section} % Формула вида секция.номер

\renewcommand{\footnotesize}{\small}


% Оформление библиографии и подрисуночных записей через точку
\makeatletter
\renewcommand*{\@biblabel}[1]{\hfill#1.}
\renewcommand*\l@section{\@dottedtocline{1}{1em}{1em}}
\renewcommand{\thefigure}{\thesection.\arabic{figure}} % Формат рисунка секция.номер
\renewcommand{\thetable}{\thesection.\arabic{table}} % Формат таблицы секция.номер
\def\redeflsection{\def\l@section{\@dottedtocline{1}{0em}{10em}}}
\makeatother

% Полуторный межстрочный интервал
%%\renewcommand{\baselinestretch}{1.4}
\setlength{\parskip}{3ex}
\setlength{\parindent}{0cm} % Абзацный отступ

\sloppy
\hyphenpenalty=1000
\clubpenalty=10000
\widowpenalty=10000

% Отступы у страниц
\usepackage{geometry}
\geometry{left=3cm}
\geometry{top=2cm}
\geometry{right=1.5cm}
\geometry{bottom=2cm}

%
% Списки
%
%\usepackage{enumerate} Если вдруг сломается что-то
\usepackage{enumitem}
% TODO: Проверить topsep vs nolistsep
\setlist[enumerate,itemize]{nolistsep,topsep=0pt,parsep=0.5cm}

%%\makeatletter
%%\AddEnumerateCounter{\asbuk}{\russian@alph}{щ}
%%\makeatother
%%\setlist{nolistsep} % Нет отступов между пунктами списка
\renewcommand{\labelitemi}{--} % Маркет списка --
%%\renewcommand{\labelenumi}{\asbuk{enumi})} % Список второго уровня
%%\renewcommand{\labelenumii}{\arabic{enumii})} % Список третьего уровня



%
% Содержание
%
\usepackage{tocloft}
\usepackage[dotinlabels]{titletoc}
\renewcommand{\cfttoctitlefont}
    {\normalfont\bfseries\hspace{0.38\textwidth}\MakeTextUppercase}
\renewcommand{\cftsecfont}{\hspace{0pt}}
% Точки для секций в содержании
%%\renewcommand\cftsecleader{\cftdotfill{\cftdotsep}}
\renewcommand\cftsecpagefont{\mdseries}
\setcounter{tocdepth}{3}



%
% Рисунки
%

\usepackage{graphicx}

% Формат подрисуночных записей
\usepackage{chngcntr}
\counterwithin{figure}{section}

% Формат подрисуночных надписей
\RequirePackage{caption}
\DeclareCaptionLabelSeparator{defffis}{ }
\captionsetup[figure]{justification=centering, labelsep=defffis, format=plain}
\captionsetup[table]{justification=raggedright, labelsep=defffis, format=plain, singlelinecheck=false}
\addto\captionsrussian{\renewcommand{\figurename}{Рис.}}

% Пользовательские функции
\newcommand{\addimg}[4]{ % Добавление одного рисунка
    \begin{figure}
        \centering
        \includegraphics[width=#2\linewidth]{#1}
        \caption{#3} \label{#4}
    \end{figure}
}
\newcommand{\addimghere}[4]{ % Добавить рисунок непосредственно в это место
    \begin{figure}[H]
        \centering
        \includegraphics[width=#2\linewidth]{#1}
        \caption{#3} \label{#4}
    \end{figure}
}
\newcommand{\addanonimghere}[2]{
    \begin{figure}[H]
        \centering
        \includegraphics[width=#2\linewidth]{#1}
    \end{figure}
}
\newcommand{\addimgherev}[5]{ % Добавить рисунок непосредственно в это место
    \begin{figure}[H]
        \centering
        \includegraphics[width=#3\linewidth]{#1}
        \includegraphics[width=#3\linewidth]{#2}
        \caption{#4} \label{#5}
    \end{figure}
}
\newcommand{\addtwoimghere}[5]{ % Вставка двух рисунков
    \begin{figure}[H]
        \centering
        \includegraphics[width=#3\linewidth]{#1}
        \hfill
        \includegraphics[width=#3\linewidth]{#2}
        \caption{#4} \label{#5}
    \end{figure}
}
\newcommand{\addimgapp}[2]{ % Это костыль для приложения Б
    \begin{figure}[H]
        \centering
        \includegraphics[width=1\linewidth]{#1}
        \caption*{#2}
    \end{figure}
}
\newcommand{\addcircform}[5]{
    \hspace*{-1cm}%
    \begin{minipage}{#4\textwidth}
        \addimghere{#1}{}{#2}{3}
        \vspace{0.3cm}
    \end{minipage}%
    \hfill%
    \begin{minipage}{#5\textwidth}
        #3
        \vspace{3.5cm}%
    \end{minipage}
}



%
% Таблицы
%
\usepackage{float}
\usepackage{tabularx}
\usepackage{longtable}
\usepackage{multirow}
\usepackage{tabu}


% Хаки для таблиц
\newcommand{\centmultirow}[3]{%
    \multicolumn{1}{|c|}{\multirow{#1}{#2}{\centering #3}}%
}
\newcolumntype{P}[1]{>{\centering\arraybackslash}p{#1}}
\newcolumntype{C}{>{\centering\arraybackslash}X}

\newcommand{\inputtabhere}[3]{%
    \begin{table}[h]
        \renewcommand{\arraystretch}{1.30}
        \renewcommand{\tabularxcolumn}[1]{m{##1}}
        \input{#1}
        \caption{#2}
        \label{#3}
    \end{table}
}



%
% Секции
%

% Заголовки секций в оглавлении в верхнем регистре
\usepackage{textcase}
\makeatletter
\let\oldcontentsline\contentsline
\def\contentsline#1#2{
    \expandafter\ifx\csname l@#1\endcsname\l@section
        \expandafter\@firstoftwo
    \else
        \expandafter\@secondoftwo
    \fi
    {\oldcontentsline{#1}{#2}}
    {\oldcontentsline{#1}{#2}}
}
\makeatother

% Оформление заголовков
\usepackage[compact,explicit]{titlesec}
\titleformat{\section}{}{}{0pt}
    {\centering\textbf{\thesection.\space\MakeTextUppercase{#1}}}
\titleformat{\subsection}[block]{\vspace{1em}}{}{0pt}
    {\centering\textbf{\thesubsection.\space#1}}
\titleformat{\subsubsection}[block]{\vspace{1em}\normalsize}{}{0pt}
    {\centering\textbf{\thesubsubsection.\space#1\vspace{1em}}}
\titleformat{\paragraph}[block] {\normalsize}{}{12.5mm}
    {\MakeTextUppercase{#1}}

% Секции без номеров (введение, заключение...), вместо section*{}
\newcommand{\anonsection}[1]{
    \vspace{1.5em}
    \phantomsection % Корректный переход по ссылкам в содержании
    \paragraph{\normalfont\bfseries\centerline{{#1}}}
    \addcontentsline{toc}{section}{#1}
}

% Секции для приложений
\newcommand{\appsection}[1]{
    \phantomsection
    \paragraph{\centerline{{#1}}}
    \addcontentsline{toc}{section}{\uppercase{#1}}
}



%
% Остальное
%

%\usepackage{listings} % Оформление исходного кода
%\lstset{
%    language=C++,
%    title=\lstname,
%    basicstyle=\small\ttfamily, % Размер и тип шрифта
%    breaklines=\true, % Перенос строк
%    tabsize=2, % Размер табуляции
%    literate={--}{{-{}-}}2, % Корректно отображать двойной дефис
%    texcl=\true,
%    showspaces=\false,
%    showstringspaces=\false,
%    extendedchars=\true,
%}

% Гиперссылки
%%
%% ВНИМАНИЕ!
%% \usepackage{hyperref} должен идти после всех остальных \usepackage.
%%
\usepackage{ulem}
\usepackage{hyperref}
\urlstyle{same}
\hypersetup{
    colorlinks, urlcolor=blue,
    linkcolor={black}, citecolor={black}, filecolor={black},
    pdfauthor={Репин Степан}, pdfborderstyle={/S/U/W 1}, allbordercolors=0 0 1,
}


% Простой текст в формулах
\newcommand{\tc}[1]{\text{#1}}
% Единицы измерения
\newcommand{\un}[1]{\tc{\ \ #1}}

% Различные физические обозначения.
\newcommand{\Lap}[0]{\mathcal{L}}
\newcommand{\ILap}[0]{\Lap^{-1}}
\newcommand{\Hev}[0]{\delta_1}

% Настройка CircuiTikz
%%\ctikzset{voltage/distance from node=1cm}
%%\ctikzset{bipoles/thickness=1}



\usepackage[outputdir=out,cache=false]{minted}

\begin{document}

\begin{titlepage}
\newgeometry{left=1.18in,top=0.60in,right=0.39in,bottom=0.59in}
\begin{center}
    \uppercase{\textbf{Минобрнауки России\\
            Санкт-Петербургский государственный\\
            электротехнический университет\\
            «ЛЭТИ» им. В.И.Ульянова (Ленина)
    }}
    \vspace{0.25cm}

    \textbf{Кафедра ВТ}
    \vfill

    \uppercase{\textbf{\large{
        Отчет
    }}}
    \\
    \textbf{\large{
      по лабораторной работе №2\\
      по дисциплине «Параллельные алгоритмы и структуры данных»\\
      Тема: Линейные разделяемые структуры данных
      \vspace{0.5cm}
    }}
  \bigskip
\end{center}
\vfill
\vfill

\noindent

\begin{tabularx}{\textwidth}{@{}lcXr}
    Студенты гр. 8307 & \hspace{1.6cm} & \rule{5cm}{1pt} & Николаев Д.E.
\end{tabularx}

\begin{tabularx}{\textwidth}{@{}lcXr}
    \phantom{Студенты гр. 8307} & \hspace{1.6cm} & \rule{5cm}{1pt} & Репин С.А.
\end{tabularx}

\vspace{0.5cm}

\noindent
\begin{tabularx}{\textwidth}{@{}lcXr}
    Преподаватель & \hspace{2.3cm} & \rule{5cm}{1pt} & Пазников А.А.
\end{tabularx}

\hfill \break
\hfill \break

\begin{center}
  Санкт-Петербург\\2022
\end{center}

\end{titlepage}



\renewcommand*{\thepage}{}
\tableofcontents
\clearpage
\renewcommand*{\thepage}{\arabic{page}}

\setcounter{page}{3}

\section{Задание}

Реализовать lockless стек на основе алгоритма Elimination back-off stack.

\section{Выполнение работы}

Алгоритм Elimination back-off stack написан на языке C++20 для ОС Linux x64 на
основе публикации \cite{hendler04}.

Elimination back-off -- структура данных типа стек, имеющая свойства
неблокируемости, линерализуемости и масштабируемости. В простом lockless стеке
операция CAS может быть очень затратной и слабо масштабируемой с ростом числа
потоков. Применение специальной back-off стратегии -- алгоритма, исполняющегося после
неудачного CAS и призванного разгрузить процессор -- позволяет устранить этот
недостаток. В данном случае в качестве такой стратегии используется специальный
массив, содержащий дескрипторы выполняемых разными потоками операций, что дает
возможность в некоторых случаях взаимно поглотить операции pop и push -- не
совершать их в реальности.

Кроме того был реализован тест, запускающий $N$ потоков, каждый из которых
выполняет push- и pop-операции над стеком, порядка десятков тысяч раз. По
завершению тестовая программа сверяет полученное состояние стека с ожидаемым.
Измеряется время с момента запуска потоков до их завершения.

На графиках ниже показана зависимость времени выполнения от числа потоков для
двух реализаций тестовой программы: с использованием Elimination back-off stack
и стека std::stack с мьютексом. Из графика следует, что с ростом contention,
когда все больше потоков соревнуются за доступ к критической секции, время
выполнения обеих реализаций сначала сравниваются, а затем EBS начинает обгонять
реализацию на мьютексе (сравниваются на 15-16 потоках, EBS обгоняет с 25). А
также кривая времени для lockless реализации намного более гладкая.
Таким образом, применение Elimination back-off stack оправдано при большом
количестве потоков.

Параметры стенда представлены в таблице ниже:

\begin{table}[H]
\centering
\begin{tabular}{|c|c|}
    \hline
    Название & Значение \\
    \hline
    ОС & Arch Linux (Linux 5.16 x64) \\
    \hline
    Процессор & Intel Core i7 11700 \\
    \hline
    Число ядер & 16 (8) \\
    \hline
\end{tabular}
\end{table}

\addimghere{res/plot2.png}{1}{График для 2-32 потока}{}
\addimghere{res/plot.png}{1}{График для 2-200 потоков}{}

\section{Разделение работы в бригаде}

Вместе выяснили и обсудили суть алгоритма, особенности его реализации.
Определились с интерфейсом библиотеки и ее тестированием.

Дмитрий Николаев выполнил разработку тестовой утилиты, скриптов для анализа
результатов. Выполнил отладку и рефакторинг. Сделал реализацию на мьютексах.

Степан Репин реализовал Elimination back-off stack.

\begingroup
\renewcommand{\section}[2]{\anonsection{Список использованных источников}}
\begin{thebibliography}{00}

\bibitem{hendler04}
    D. Hendler, N. Shavit, L. Yerushalmi. A Scalable Lock-free Stack Algorithm
    // In Proceedings of the sixteenth annual ACM symposium on Parallelism in
    algorithms and architectures (SPAA '04). Association for Computing
    Machinery. - 2004. - P. 206–215.

\end{thebibliography}
\endgroup

\end{document}

