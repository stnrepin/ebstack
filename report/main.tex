\documentclass[a4paper,14pt]{extarticle}
\usepackage{rotating}
\usepackage{verbatim}
\input{inc/preamble.tex}

\usepackage[outputdir=out,cache=false]{minted}

\begin{document}

\input{inc/titlepage}

\renewcommand*{\thepage}{}
\tableofcontents
\clearpage
\renewcommand*{\thepage}{\arabic{page}}

\setcounter{page}{3}

\section{Задание}

Реализовать lockless стек на основе алгоритма Elimination back-off stack.

\section{Выполнение работы}

Алгоритм Elimination back-off stack написан на языке C++20 для ОС Linux x64 на
основе публикации \cite{hendler04}.

Elimination back-off -- структура данных типа стек, имеющая свойства
неблокируемости, линерализуемости и масштабируемости. В простом lockless стеке
операция CAS может быть очень затратной и слабо масштабируемой с ростом числа
потоков. Применение специальной back-off стратегии -- алгоритма, исполняющегося после
неудачного CAS и призванного разгрузить процессор -- позволяет устранить этот
недостаток. В данном случае в качестве такой стратегии используется специальный
массив, содержащий дескрипторы выполняемых разными потоками операций, что дает
возможность в некоторых случаях взаимно поглотить операции pop и push -- не
совершать их в реальности.

Кроме того был реализован тест, запускающий $N$ потоков, каждый из которых
выполняет push- и pop-операции над стеком, порядка десятков тысяч раз. По
завершению тестовая программа сверяет полученное состояние стека с ожидаемым.
Измеряется время с момента запуска потоков до их завершения.

На графиках ниже показана зависимость времени выполнения от числа потоков для
двух реализаций тестовой программы: с использованием Elimination back-off stack
и стека std::stack с мьютексом. Из графика следует, что с ростом contention,
когда все больше потоков соревнуются за доступ к критической секции, время
выполнения обеих реализаций сначала сравниваются, а затем EBS начинает обгонять
реализацию на мьютексе (сравниваются на 15-16 потоках, EBS обгоняет с 25). А
также кривая времени для lockless реализации намного более гладкая.
Таким образом, применение Elimination back-off stack оправдано при большом
количестве потоков.

Параметры стенда представлены в таблице ниже:

\begin{table}[H]
\centering
\begin{tabular}{|c|c|}
    \hline
    Название & Значение \\
    \hline
    ОС & Arch Linux (Linux 5.16 x64) \\
    \hline
    Процессор & Intel Core i7 11700 \\
    \hline
    Число ядер & 16 (8) \\
    \hline
\end{tabular}
\end{table}

\addimghere{res/plot2.png}{1}{График для 2-32 потока}{}
\addimghere{res/plot.png}{1}{График для 2-200 потоков}{}

\section{Разделение работы в бригаде}

Вместе выяснили и обсудили суть алгоритма, особенности его реализации.
Определились с интерфейсом библиотеки и ее тестированием.

Дмитрий Николаев выполнил разработку тестовой утилиты, скриптов для анализа
результатов. Выполнил отладку и рефакторинг. Сделал реализацию на мьютексах.

Степан Репин реализовал Elimination back-off stack.

\begingroup
\renewcommand{\section}[2]{\anonsection{Список использованных источников}}
\begin{thebibliography}{00}

\bibitem{hendler04}
    D. Hendler, N. Shavit, L. Yerushalmi. A Scalable Lock-free Stack Algorithm
    // In Proceedings of the sixteenth annual ACM symposium on Parallelism in
    algorithms and architectures (SPAA '04). Association for Computing
    Machinery. - 2004. - P. 206–215.

\end{thebibliography}
\endgroup

\end{document}

